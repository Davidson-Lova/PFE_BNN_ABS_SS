\documentclass{article}

% Language setting
% Replace `english' with e.g. `spanish' to change the document language
\usepackage[english]{babel}

% Set page size and margins
% Replace `letterpaper' with `a4paper' for UK/EU standard size
\usepackage[letterpaper,top=2cm,bottom=2cm,left=3cm,right=3cm,marginparwidth=1.75cm]{geometry}

% Useful packages
\usepackage{amsmath}
\usepackage{graphicx}
\usepackage[colorlinks=true, allcolors=blue]{hyperref}

\title{PFE}
\author{Khaled MEDJKOUH \\ Davidson Lova RAZAFINDRAKOTO}

\begin{document}

\maketitle

\pagebreak

\tableofcontents

\pagebreak

\section{Introduction}
Incertitudes utile dans plusieurs domaines et la quantification duquel
est nécessaire pour des évaluations de risques, déctection d'anomalie et prise décision.


\pagebreak

\section{Réseau de neurones}

+ Historique sur les réseaux de neurones

+ Explication des poids et biais et fonction d'activation

+ Quelque résultats et application

+ Ce qui nous amène à un problème


\pagebreak

\section{Réseau de neurones bayésiens}

+ Présentation de réseaux de neurones bayésiens

+ Donner son interet dans la résolution du problème précédent

+ Donner les méthodes d'entrainement d'un tel réseaux

\pagebreak

\section{Algorithme BNN-ABC-SS}

\subsection{ABC}

+ celui dans le rapport d'avant

\subsection{SS}

+ celui dans le rapport d'avant

\subsection{Finalité}

+ celui dans le rapprot d'avant

\pagebreak

\section{Réalisation}

+ On se donne un base de données à étudier
avec une comparasion avec des méthodes déja existantes

\end{document}